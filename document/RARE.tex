The \dsl language\footnote{\dsl comes from {\sc R}ewrites, {\sc a}utomatically
{\sc re}constructed.} was introduced by N\"otzli et al.~\cite{notzli2022reconstructing}.
As part of this work, we have extended the language to be able to
represent more rewrite rules.  We present the full updated language here and
summarize the differences with~\cite{notzli2022reconstructing} at the end of the
section.

A \dsl file contains a list of rules whose syntax is defined by the grammar in Figure \ref{fig:grammar}.
Expressions use SMT-LIB syntax with a few exceptions.
These include the use of approximate sorts for parameterized sorts (e.g.,
arrays and bit-vectors) and the addition of a few extra operators (e.g.,
\rareInline{bvsize}, described below). \dsl uses SMT-LIB 3
syntax~\cite{smtlib3}, which is very close to SMT-LIB 2 and mostly differs from
its predecessor in that it uses higher-order functions for indexed operators. 
%There are other small syntax changes, for example bit-vector constants now explicitly include their bit-width.
%I think this is not an SMT-LIB change (yet?), either find other example or leave away -> Asked on zulip and this will not be included in SMT-LIB 3

\begin{figure}[t]
    \grammarindent0.8in
    \renewcommand{\ulitleft}{\small \ttfamily \bfseries}
    \renewcommand{\ulitright}{}
    \begin{grammar}
      <rule> ::= "(" "define-rule" <symbol> "(" <par>$^*$ ")" [<defs>] <expr> <expr> ")"
      \alt "(" "define-rule*" <symbol> "(" <par>$^*$ ")" [<defs>] <expr> <expr> [<expr>] ")"
      \alt "(" "define-cond-rule" <symbol> "(" <par>$^*$ ")" [<defs>] <expr> <expr> <expr> ")"

      <par> ::= <symbol> <sort> [":list"]

      <sort> ::= <symbol> | "?" | "?"<symbol> | "(" <symbol> <numeral>$^+$ ")"

      <expr> ::= <const> | <id> | "(" <id> <expr>$^+$")"

      <id> ::= <symbol> | "(" <symbol> <numeral>$^+$ ")"

      <binding> ::= "(" <symbol> <expr> ")"

      <defs> ::= "(" "def" <binding>$^+$ ")"
    \end{grammar}

    \caption{Overview of the grammar of \dsl.}
    \label{fig:grammar}
\end{figure}

We say that an expression $e$ \emph{matches}
a match expression $m$ if there is some \emph{matching substitution} $\sigma$
that replaces each variable in $m$ by a term of the same sort to obtain $e$ 
(i.e., $m\sigma$ is syntactically identical to $e$).
For example, the expression \rareInline{(or (bvugt x1 x2) (= x2 x3))},
with variables \rareInline{x1}, \rareInline{x2}, \rareInline{x3},
all of sort \rareInline{?BitVec},
matches \rareInline{(or (bvugt a b) (= b a))}
but not \rareInline{(or (bvugt a b) (= c a))},
with \rareInline{a}, \rareInline{b}, and \rareInline{c} bit-vector constant
symbols of the same bit-width.

\subsubsection{\dsl Rules}
A \dsl rewrite rule is defined with the \rareInline{define-rule} command
which starts with a parameter list containing variables with their sorts.  
These variables are used for matching as explained below.  
After an optional \emph{definition list} (see below), 
there follow two expressions that form the main body of the rule:
the \emph{match} expression and the \emph{target} expression.
The semantics of a rule with match expression $m$ and target expression $t$
is that any expression $e$ matching $m$ under some sort-preserving matching substitution $\sigma$
can be replaced by $t\sigma$.
%
With approximate sorts, the sort preservation requirement is relaxed as follows.
In \dsl, for any sort constructor $S$ of arity $n > 0$, 
there is a corresponding approximate sort \rareInline{($S$ ?\ $\cdots$\ ?)}
with $n$ occurrences of \rareInline{?}
which is always abbreviated as \rareInline{?$S$}.
A variable $x$ with sort \rareInline{?$S$} (e.g., \rareInline{?BitVec}) 
in a match expression matches all expressions whose sort is constructed with $S$
(e.g., \rareInline{(BitVec 1)}, \rareInline{(BitVec 2),} and so on).
Variables with sort \rareInline{?} match expressions of any sort.

An optional \textit{definition list} may appear in a \dsl rule immediately 
after the parameter list. 
It starts with the keyword \rareInline{def} and provides 
a list of local variables and their definitions,
allowing the rewrite rule to be expressed more succinctly.
A rule with a definition list is equivalent to the same
rule without it, where each variable in the definition list
has been replaced by its corresponding expression in the body of the rule.
For a rule to be well-formed, all variables in the match and target
expressions must appear either in the parameter list or the definition list.
Similarly, each variable in the parameter list must appear in the match
expression (while this second requirement could be relaxed, it is useful for
catching mistakes).
%
Consider the following example.

\begin{lstlisting}[language=rare]
(define-rule bv-sign-extend-eliminate ((x ?BitVec) (n Int))
  (def (s (bvsize x)))
  (sign_extend n x)   (concat (repeat n (extract (- s 1) (- s 1) x)) x))
\end{lstlisting}

\noindent
In this rule, there are two parameters, \rareInline{x}
and \rareInline{n}.
The sort annotation \rareInline{?BitVec} indicates that \rareInline{x}
is a bit-vector without specifying its bit-width.
%
The latter is stored in the local variable \rareInline{s} 
using the \rareInline{bvsize} operator.  
The rule says that a \rareInline{(sign_extend n x)} expression can be replaced
by repeating \rareInline{n} times the most significant bit of \rareInline{x}
and then prepending this to \rareInline{x}.

The \rareInline{define-cond-rule} command is similar to \rareInline{define-rule} except that
it has an additional expression, the \textit{condition}, immediately after the
parameter and definition lists. This restricts the rule's applicability
to cases where the condition can be proven equivalent to true under the
matching substitution. 
In the example below, the condition \rareInline{(> n 1)} can be verified by evaluation 
since in SMT-LIB, the first argument of \rareInline{repeat} must be a numeral.  

\begin{lstlisting}[language=rare]
(define-cond-rule bv-repeat-eliminate-1 ((x ?BitVec) (n Int))
  (> n 1)   (repeat n x)   (concat x (repeat (- n 1) x)))
\end{lstlisting}
%
%%CT This explanation is not really needed
%This rule states that the first argument of a \rareInline{repeat} operation can
%be decremented by one by introducing a \rareInline{concat} operation.
Note that the rule does not apply to terms like \rareInline{(repeat 1\ $t$)} 
or \rareInline{(repeat 0\ $t$)}.

\subsubsection{Fixed-point Rules}

The \rareInline{define-rule*} command defines rules that
%, during elaboration, 
should be applied repeatedly, to completion.
%
This is useful, for instance, in writing rules that iterate over
the arguments of n-ary operators.  Its basic form, 
with a body containing just a match and target expression, 
defines a rule that, whenever is applied, must be applied again 
on the resulting term until it no longer applies.

The user can optionally supply a \emph{context} to control the iteration.  This
is a third expression that must contain an underscore.  
The semantics is that the match expression rewrites to the context expression, with
the underscore replaced by the target expression.  Then the rule is applied
again to the target expression only. 
In the example below,
the \rareInline{:list} modifier is used to represent an arbitrary number
of arguments, including zero, of the same type.  

\begin{lstlisting}[language=rare]
(define-rule* bv-neg-add ((x ?BitVec) (y ?BitVec) (zs ?BitVec :list))
  (bvneg (bvadd x y zs))   (bvneg (bvadd y zs))   (bvadd (bvneg x) _))
\end{lstlisting}
%
This rule rewrites a term \rareInline{(bvneg (bvadd\ $s \ t \ \cdots$))}
to the term \rareInline{(bvadd (bvneg\ $s$)\ $r$)} 
where $r$ is the result of recursively applying the rule 
to \rareInline{(bvneg (bvadd\ $t \ \cdots$))}.

\subsubsection{Changes to \dsl}
Here, we briefly mention the changes to \dsl with respect
to \cite{notzli2022reconstructing}.  First, we have support for a richer class
of approximate sorts, including approximate bit-vector and array sorts.  Also, we
replaced the \rareInline{let} construct by the new \rareInline{def} construct.
The definition list is more powerful as it applies to the entire rest of the
body (whereas \rareInline{let} was local to a single expression).

Additionally, to aid with bit-vector rewrite rules, we added several operators:
\rareInline{bvsize}, which returns the width of an expression of sort
\rareInline{?BitVec}; \rareInline{bv}, which takes a integer $n$ and natural $w$, 
and returns a bit-vector of width $w$ and value $n\ \mathrm{mod}\ 2^w$;
\rareInline{int.log2} which returns the integer (base 2) logarithm of an integer,
and \rareInline{int.islog2}, which returns true iff its integer argument is a power of 2.

We also removed the \rareInline{:const} modifier, which was used
previously to indicate that a particular expression had to be a constant
value.  We found that this adds complexity and is usually unnecessary.  For rules that actually manipulate specific constant values,
we can specify those values explicitly, e.g., by using
the \rareInline{bv} operator above.

